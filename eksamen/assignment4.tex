\documentclass{article}
\usepackage[danish]{babel}
\usepackage{graphicx}
\usepackage{outlines}
\usepackage{amssymb}
\usepackage{enumitem}
\usepackage{listings}
\usepackage{tabularx}
\usepackage{multirow}
\usepackage{listings}
\usepackage[T1]{fontenc}
\usepackage{inconsolata}
\usepackage{color}
\usepackage{hyperref}
\renewcommand{\labelitemi}{$\bullet$}
\renewcommand{\labelitemii}{$\circ$}
\renewcommand{\labelitemiii}{$\blacksquare$}
\renewcommand{\labelitemiv}{$\star$}
\definecolor{pblue}{rgb}{0.13,0.13,1}
\definecolor{pgreen}{rgb}{0,0.5,0}
\definecolor{pred}{rgb}{0.9,0,0}
\definecolor{pgrey}{rgb}{0.46,0.45,0.48}
\lstset{language=Java,
  showspaces=false,
  showtabs=false,
  breaklines=true,
  showstringspaces=false,
  breakatwhitespace=true,
  commentstyle=\color{pgreen},
  keywordstyle=\color{pblue},
  stringstyle=\color{pred},
  basicstyle=\ttfamily,
  moredelim=[il][\textcolor{pgrey}]{$$},
  moredelim=[is][\textcolor{pgrey}]{\%\%}{\%\%}
}
\title{%
Bremses udviklingen af sin fart?
}
\author{Mads Wulff Nielsen \\
Claus Kramath}
\begin{document}
\maketitle
\thispagestyle{empty}
\newpage
\tableofcontents
\thispagestyle{empty} 
\newpage
\section{Forord}
\paragraph{}
Det smarte hjem er i sin vugge, men går teknologiudviklingen så
stærkt, at den bremses? Vi ser på hvilke mekanismer der er i spil, når nye
landevindinger går hurtigere, end mennesket måske er klar til, med fokus
rettet mod it-teknologi i særdeleshed.
\section{Indledning}
\paragraph{}
Dele af neurovidenskaben argumenterer for, at vores hjerne ikke er fulgt med den øvrige udvikling,
og således anser ændringer i tillærte vaner for farlige \footnote{\url{https://www.youtube.com/watch?v=JX3RkHtFP4Q}}. Man taler eksempelvis om loss-of-control 
\footnote{\url{https://www.psychologytoday.com/us/blog/what-would-aristotle-do/201105/the-fear-losing-control}} når nye ting introduceres eller skal indlæres \footnote{\url{https://digitalwellbeing.org/are-our-brains-really-no-match-for-our-technology/} \footnote{https://www.youtube.com/watch?v=JX3RkHtFP4Q}}. \\
Samtidig har mennesket gennem tusinder af år skabt civilisationer og foretaget landevindinger og har på den måde udviklet sig til trods for ovenstående påstand.
\paragraph{}
Nye teknologiers fremmarch går ofte hånd i hånd med skepsis \footnote{\url{https://www.verywellmind.com/what-is-the-fear-of-technology-2671897}}.
Dette begreb; teknofobi, kender vi fra tidligere tiders opfindelser, såsom: den førerløse elevator \footnote{\url{https://www.npr.org/2015/07/31/427990392/remembering-when-driverless-elevators-drew-skepticism}}, mikrobølgeovnen \footnote{\url{http://historyoftech.mcclurken.org/microwave/the-impact/}}
samt aktuelt den førerløse bil \footnote{\url{https://www.nal.com/blog/fear-of-autonomous-vehicles/}}
Da der i det menneskeskabte samfund, udover den før omtalte påstand, også ofte er mange meninger og hensyn at tage, og da teknologier kan være svære at forstå omfanget og følgerne af, er der således modsatrettede kræfter igang, når ny teknologi dukker op.
\paragraph{}
Vi vil kigge nærmere på årsagerne til at it-teknologiudviklingens høje hastighed også kan være det, der bremser den, med særligt fokus på intelligente hjem.
\section{Metode}
\paragraph{}
For at kunne afgøre, om der reelt er tale om en bremsende effekt, vil vi fortolke 
data som forudsiger, hvordan udbredelsen af intelligente hjem burde se ud nu og 
sammenligne disse med den faktiske tilstand.
Vi vil også kigge på historiske data om udbredelsen af mikrobølgeovnen, tv, biler, kloning mv. samt indsamle data vedr. folks holdning til it-teknologi på baggrund af den seneste tids omtale af blandt andet påvirkningen af det amerikanske valg i 2016, førerløse biler, noget andet godt og så videre.
https://www.the-ambient.com/features/future-of-smart-home-timeline-310 , https://guidehouseinsights.com/news-and-views/smart-home-predictions-what-will-and-wont-happen-in-2019 , https://visual.ly/community/Infographics/technology/smart-homes-evolution-timeline , 
\section{Historisk udvikling}
\paragraph{}
Udviklingen af salget af mikrobølgeovne tog fart i 1980'erne, især i USA, hvor 25\% af de amerikanske husholdninger havde en mikrobølgeovn i 1986 \footnote{\url{https://www.timetoast.com/timelines/the-history-of-a-microwave-oven}}.
Til sammenligning var det i 1971 blot 1\%.
\begin{figure}[htb]
    \centering
    \includegraphics[width=\textwidth]{microwaveovens.PNG}
    \caption{Der skulle gå godt 47 år fra opfindelse til udbredelse af mikrobølgeovnen.}
    \label{fig:microwaveovens}
\end{figure}
\paragraph{}
På samme måde kan man i dag opleve tilbageholdenhed i forhold til førerløse biler. Et studie fra AAA, den amerikanske pendant
til FDM i Danmark, foretaget i 2019, viser, at 3 ud af 4 amerikanere er bange for den fuldt selvkørende bil.
Dette tal er en stigning i forhold til tallet fra en lignende undersøgelse fra 2017, her var andelen blot 63\% \footnote{\url{https://www.forbes.com/sites/tanyamohn/2019/03/28/most-americans-still-afraid-to-ride-in-self-driving-cars/?sh=205ca6ec32da}}.
\section{Intelligente hjem}
\paragraph{}
Selvom udviklingen af enheder til intelligente hjem har været igang i en årrække, synes teknologien som helhed stadig at være infantil.
Som netop beskrevet, skal en af årsagerne også her findes i, at der hersker en tydelig skepsis blandt potentielle forbrugere \footnote{\url{https://www.businessinsider.com/consumers-holding-off-on-smart-home-gadgets-thanks-to-privacy-fears-2017-11?r=US\&IR=T}}.
Denne skepsis, som generelt kan betragtes som en frygt for det ukendte, kan dog blandt andet tilskrives følgende årsager:
\begin{enumerate}
    \item manglende lovgivning, f.eks. i forbindelse med overvågning, anonymisering af data, sikkerhed, ansvar (f.eks. selvkørende biler, når de fejler, hvis er skylden så?)
    \item Den juridiske proces er ofte langsommelig ift. teknologien; når en beslutning er truffet i en lovgivende forsamling, er den teknologi, der blev lovgivet om, forandret \footnote{url{https://www.weforum.org/agenda/2018/06/law-too-slow-for-new-tech-how-keep-up/}}
    \item manglende hjælp når der opstår problemer, da teknologien er så ny, er det svært at få hjælp hvis man oplever problemer (hacking, fraud, exploitation osv.)
    \item mangel på standarder og protokoller gør betjening af mange enheder besværligt \footnote{\url{https://www.euractiv.com/section/energy/opinion/are-we-there-yet-current-state-of-the-smart-home-market/}}
\end{enumerate} 
Interessante betragtninger: Lovgivere tvinger, måske uforvarende, teknologi ned over befolkningen, uden at være klar over hvilke, om muligt, uønskede teknologier, befolkningen får med i købet.
Omvendt er lovgiverne tilbageholdende med at lempe lovgivningen således at relevante myndigheder kan udvikle værktøjer som lettere kan afdække kriminelle forhold, økonomiske uregelmæssigheder mv. (kan vi få en kilde her?)
Og så var der den om de nuværende, mindre kritiske unge, der snart skal til at etablere sig i eget hjem, vil de forholde sig lige så ukritisk eller vil der begynde at indfinde sig en reel skepsis - opfører man sig anderledes når man ved, enhederne lytter med?

\section{Konklusion}
\paragraph{}
Baseret på de kilder vi har lagt til grund for vores artikel, må vi konkludere, at teknologiens udviklingshastighed også bremser den, dog i overført betydning. For ny teknologi har en ramme, den skal fungere i. En ramme der ofte er forbundet med lovgivning og samfundsmæssige hensyn. Selvom det tilsyneladende er sådan, at det er rammerne der forholder sig til teknologien og ikke omvendt, ender det oftest med at teknologien indordnes i en tilrettet version 2. (kilde)

\section{Næste skridt (hedder det det?)}
Artiklen er overvejende baseret på internationale data, læses den med et regionalt udgangspunkt, bør den suppleres med data fra den lokale befolkning. Vi formoder at der er demografiske forhold som vil være sporbare i disse. Eksempelvis kan man forestille sig, at man i takt med den stigende overvågning af Kinas befolkning, vil kunne finde andre, interessante svar, hvis undersøgelser blev foretaget blandt disse individer - dette fordrer dog at den frie tale er mulig... 
Man kunne f.eks. spørge:
\begin{enumerate}
    \item Hvad synes du om at data om dig gemmes?
    \item Ved du hvordan data anonymiseres?
    \item Har du tænkt over, om du kan tale fortroligt derhjemme, når telefonen formentlig lytter med?
    \item Listes teknologier, som overvåger dig, ind i dit liv, uden du er klar over det?
    \item osvosv.    
\end{enumerate}
\newpage
\section{TBD}
Er der funktionalitet du ikke vil undvære, og dermed giver du køb på dine holdninger?
Ved du hvad det vil sige, når nogle firmaer "lytter med"?
Coronapas/nemid
Teknologi
? er om politikerne er klar over hvad der trækkes ned over hovedet på folk?
Eller - listes teknologi ind, som vi, havde vi haft et oplyst grundlag, bevidst havde fravalgt?


Hvad er den afledte effekt af udviklingens hastighed?
Større/mindre skepsis overfor nye teknologier?
Er folk bevidste om konsekvenserne af deres valg?
Har folk et valg? 
- kan jeg gå på restaurant uden smartphone for tiden?
- kan jeg overhovedet betale for noget uden smartphone?
- 

Smart homes er måske også mere minded for den mere realistiske forbruger (dvs 35+ måske med eget hjem) som træffer mere oplyste valg end hvalpen, der er 20-25 og våd bag ørerne.
Eller snarere - dem der i dag foretager ukritiske valg skal om 5 år til at investere i egen ejendom, og så tager markedet fart.
OG: Smart homes er stadig i sin vugge fordi der ikke er enighed blandt udbydere af devices ( og det stod der om i en artikel et sted, måske den med tidslinjen) om hvilken måde der skal kommunikeres på, forbindes på osv. En afledning kunne være: Der er mange udbydere af smart home devices fordi... udviklingen af de boards, der skal bruges i de devices, er gået hurtigt. Altså bremses farten i den ene udvikling, af farten i den anden udvikling...
\end{document}
\documentclass{article}
\usepackage[danish]{babel}
\usepackage{graphicx}
\usepackage{outlines}
\usepackage{amssymb}
\usepackage{enumitem}
\usepackage{listings}
\usepackage{tabularx}
\usepackage{multirow}
\usepackage{listings}
\usepackage[T1]{fontenc}
\usepackage{inconsolata}
\usepackage{color}
\usepackage{hyperref}
\renewcommand{\labelitemi}{$\bullet$}
\renewcommand{\labelitemii}{$\circ$}
\renewcommand{\labelitemiii}{$\blacksquare$}
\renewcommand{\labelitemiv}{$\star$}
\definecolor{pblue}{rgb}{0.13,0.13,1}
\definecolor{pgreen}{rgb}{0,0.5,0}
\definecolor{pred}{rgb}{0.9,0,0}
\definecolor{pgrey}{rgb}{0.46,0.45,0.48}
\lstset{language=Java,
  showspaces=false,
  showtabs=false,
  breaklines=true,
  showstringspaces=false,
  breakatwhitespace=true,
  commentstyle=\color{pgreen},
  keywordstyle=\color{pblue},
  stringstyle=\color{pred},
  basicstyle=\ttfamily,
  moredelim=[il][\textcolor{pgrey}]{$$},
  moredelim=[is][\textcolor{pgrey}]{\%\%}{\%\%}
}
\title{%
Bremses udviklingen af sin fart?
}
\author{Mads Wulff Nielsen \\
Claus Kramath}
\begin{document}
\maketitle
\thispagestyle{empty}
\newpage
\tableofcontents
\thispagestyle{empty} 
\newpage
\section{Forord}
\paragraph{}
Det smarte hjem er i sin vorden, men går teknologiudviklingen så
stærkt, at den bremses? Vi ser på hvilke mekanismer der er i spil, når nye
landevindinger går hurtigere, end mennesket måske er klar til, med fokus
rettet mod it-teknologi i særdeleshed.
\section{Indledning}
\paragraph{}
Dele af neurovidenskaben argumenterer for, at vores hjerne ikke er fulgt med den øvrige udvikling,
og således anser ændringer i tillærte vaner for farlige\footnote{\url{https://www.youtube.com/watch?v=JX3RkHtFP4Q}}. Man taler eksempelvis om loss-of-control 
\footnote{\url{https://www.psychologytoday.com/us/blog/what-would-aristotle-do/201105/the-fear-losing-control}} når nye ting introduceres eller skal indlæres\footnote{\url{https://digitalwellbeing.org/are-our-brains-really-no-match-for-our-technology/}\footnote{https://www.youtube.com/watch?v=JX3RkHtFP4Q}}. \\
Samtidig har mennesket gennem tusinder af år skabt civilisationer og foretaget landevindinger og har på den måde udviklet sig til trods for ovenstående påstand.
\paragraph{}
Nye teknologiers fremmarch går ofte hånd i hånd med skepsis\footnote{\url{https://www.verywellmind.com/what-is-the-fear-of-technology-2671897}}.
Dette begreb; teknofobi, kender vi fra tidligere tiders opfindelser, såsom: den førerløse elevator\footnote{\url{https://www.npr.org/2015/07/31/427990392/remembering-when-driverless-elevators-drew-skepticism}}, mikrobølgeovnen\footnote{\url{http://historyoftech.mcclurken.org/microwave/the-impact/}}
samt aktuelt den førerløse bil\footnote{\url{https://www.nal.com/blog/fear-of-autonomous-vehicles/}}
Da der i det menneskeskabte samfund, udover den før omtalte påstand, også ofte er mange meninger og hensyn at tage, og da teknologier kan være svære at forstå omfanget og følgerne af, er der således modsatrettede kræfter igang, når ny teknologi dukker op.
\paragraph{}
Vi vil kigge nærmere på årsagerne til at it-teknologiudviklingens høje hastighed også kan være det, der bremser den, med særligt fokus på intelligente hjem.
\section{Metode}
\paragraph{}
For at kunne afgøre, om der reelt er tale om en bremsende effekt, vil vi finde årsagerne til at forbrugere evt. er tilbageholdne med at investere i enheder til det intelligente hjem.
Og vi vil indsamle data vedr. folks holdning, og evt. ændring heraf, til it-teknologi anvendt i intelligente hjem for at fastslå om der er en sammenhæng.\\
Først vil vi dog kigge på historiske data om udbredelsen af teknologi i form af mikrobølgeovnen og tv'et, samt udviklingen af den føreløse bil, for at se om der kan drages paralleller til tidligere tiders teknologiske landevindinger.
\section{Historisk udvikling}
\subsection{Mikrobølgeovnen}
\paragraph{}
Udviklingen af salget af mikrobølgeovne tog fart i 1980'erne, især i USA, hvor 25\% af de amerikanske husholdninger havde en mikrobølgeovn i 1986\footnote{\url{https://www.timetoast.com/timelines/the-history-of-a-microwave-oven}}.
Til sammenligning var det i 1971 blot 1\%.
\begin{figure}[htb]
    \centering
    \includegraphics[width=\textwidth]{microwaveovens.PNG}
    \caption{Der skulle gå godt 47 år fra opfindelse til udbredelse af mikrobølgeovnen.}
    \label{fig:microwaveovens}
\end{figure}
\subsection{TV'et}
\paragraph{}
Selvom vi ikke har kunnet finde gode data at fundere vores påstand på, vil vi stadig mene, at udbredelsen af TV også havde en svær start. Her kan vi ikke konkludere, at mennesket var tilbageholdende på baggrund af ny teknologi, men snarere, at der var historiske rammer, der besværliggjorde udbredelsen\footnote{\url{https://courses.lumenlearning.com/suny-massmedia/chapter/9-1-the-evolution-of-television/}}. 
Årsagen til tilbageholdenhed skal snarere findes i prisen; et TV-apparat kostede i 1930'erne ca. en halv årsløn for den gennemsnitlige amerikaner. \\
Teknisk skulle der gå ca. 11 år fra salget af de første TV-apparater til etablering af tekniske standarder for transmission af signaler, hvilket betød, at man inden da måtte have et TV-apparat pr. kanal man ville se.
Samtidig udbrød 2. Verdenskrig, hvilket betød, at produktionen af TV-apparater nedlagdes til fordel for produktionen af krigsmateriel. Samtidig indskrænkedes sendefladerne. 
\subsection{Den førerløse bil}
\paragraph{}
Som tilfældet var med mikrobølgeovnen, kan man, i dag, på samme måde opleve tilbageholdenhed i forhold til førerløse biler. Et studie fra AAA, den amerikanske pendant
til FDM i Danmark, foretaget i 2019, viser, at 3 ud af 4 amerikanere er bange for den fuldt selvkørende bil.
Dette tal er en stigning i forhold til tallet fra en lignende undersøgelse fra 2017, her var andelen blot 63\%\footnote{\url{https://www.forbes.com/sites/tanyamohn/2019/03/28/most-americans-still-afraid-to-ride-in-self-driving-cars/?sh=205ca6ec32da}}.
\section{Intelligente hjem}
\subsection{Nuværende tilstand}
\paragraph{}
Selvom udviklingen af enheder til intelligente hjem har været igang i en årrække, synes teknologien som helhed stadig at være infantil.
Som netop beskrevet, skal en af årsagerne også her findes i, at der hersker en tydelig skepsis blandt potentielle forbrugere\footnote{\url{https://www.businessinsider.com/consumers-holding-off-on-smart-home-gadgets-thanks-to-privacy-fears-2017-11?r=US\&IR=T}}.\\
Denne skepsis, som generelt kan betragtes som en frygt for det ukendte, kan dog blandt andet tilskrives følgende årsager:
\begin{enumerate}
    \item Manglende lovgivning, f.eks. i forbindelse med overvågning, anonymisering af data, sikkerhed, ansvar (f.eks. når selvkørende biler fejler, hvor skal ansvaret da placeres?)
    \item Den juridiske proces er ofte langsommelig ift. teknologien; når en beslutning er truffet i en lovgivende forsamling, er den teknologi, der blev lovgivet om, forandret\footnote{\url{https://www.weforum.org/agenda/2018/06/law-too-slow-for-new-tech-how-keep-up/}}.
    \item Manglende hjælp når der opstår problemer, da teknologien er så ny, er det svært at få hjælp hvis man oplever problemer (hacking, fraud, exploitation osv.)
    \item Mangel på standarder og protokoller gør betjening af mange enheder besværligt\footnote{\url{https://www.euractiv.com/section/energy/opinion/are-we-there-yet-current-state-of-the-smart-home-market/}}.
    \item Gentagne beretninger om it-skandaler eller læk, hvor data er blevet kompromitteret eller lignende.
\end{enumerate} 
\subsection{Fremdrift trods opfattet stilstand}
\paragraph{}
Der er spor der tydeligt peger i retning af vækst, idet flere store forhandlere rapporterer om et decideret boom i salget af enheder\footnote{\url{https://www.dr.dk/nyheder/indland/intelligente-hjem-boomer-skal-vi-i-spabad-siger-vi-hey-google-aktiver-spastemning\#!/}}.
En undersøgelse fra 2020, foretaget af Danmarks Statistik\footnote{\url{https://www.dst.dk/da/Statistik/nyt/NytHtml?cid=41576}}, viser dog, at hovedårsagen til den mindre udbredelse i Danmark, skal findes i et manglende behov. Hele 58\% af dem, der fravælger smart-home teknologi angiver et manglende behov som årsag. \\
Samme årsag gjorde sig gældende i en undersøgelse fra 2016 af engelske forbrugere\footnote{\url{https://www2.deloitte.com/content/dam/Deloitte/uk/Documents/consumer-business/deloitte-uk-consumer-review-16.pdf}, figur 2}. Begge undersøgelser viser også, at tilbageholdenheden også handler om at teknologien fortsat er i udvikling, bekymring om privatlivets fred og datasikkerhed. Med andre ord, forstår forbrugerne ikke teknologien til fulde og er usikre på dens konsekvenser. Dens udvikling er gået for stærkt.\\
\paragraph{}
Den danske undersøgelse viser iøvrigt, at der er relativ ens udbredelse af forbrugere af enheder til intelligente hjem på tværs af aldersgrupperne. En interessant betragtning finder vi i det faktum, at der er næsten lige stor andel blandt grupperne 45-54 år og 16-24 år med hhv. 41\% og 43\%. \\
De unge er vokset op med alskens IT-isenkram og er vant til at bruge, forbruge og måske endda forvente forskellige services. Nogle vil måske tilføje, at brugen tendenserer at være ukritisk, dette kunne forklare den høje andel i undersøgelsen.\\
Det overraskende finder vi i andelen blandt den ældre gruppe. Vi forestiller os at de, der tilhører denne, har kendt til en verden uden internet og dermed risiko for overvågning, hacking mv., disse har vi derfor tidligere anset for værende ganske kritiske overfor intelligente hjem. \\
Tallene nævner ikke hvilke enheder, de forskellige aldersgrupper benytter sig af. Man kunne fristes til at gætte på, om den ældre gruppe har en overrepræsentation af robotstøvsugere og -græsslåmaskiner, men det må stå hen i det uvisse.
\section{Konklusion}
\paragraph{}
Baseret på de kilder vi har lagt til grund for vores artikel, må vi konkludere, at teknologiens udviklingshastighed også bremser den, dog i overført betydning. Ny teknologi har en ramme, den skal fungere i, en ramme der ofte er forbundet med lovgivning og samfundsmæssige hensyn. \\
Selvom det tilsyneladende er sådan, at det er rammerne der forholder sig til teknologien, når den kommer, og ikke teknologien der forholder sig til rammerne, mens den udvikles, ender det oftest med at teknologien indordner sig i en anden version\footnote{\url{https://www.cnbc.com/2020/11/05/digital-services-act-how-the-eu-is-going-after-big-tech.html}}.
Undertiden hører man om juridiske slagsmål forud for en eventuel tilpasning\footnote{\url{https://www.theguardian.com/technology/2019/mar/17/the-cambridge-analytica-scandal-changed-the-world-but-it-didnt-change-facebook}}, dette virker ikke befordrende på tilliden hos forbrugerne.

\section{Det videre arbejde}
Artiklen er overvejende baseret på internationale data, læses den med et regionalt udgangspunkt, bør den suppleres med data fra den lokale befolkning. \\
Vi formoder at der er demografiske forhold som vil være sporbare i disse, f.eks. kan man forestille sig, at mens europæeren er mindre bekymret over en given overvågnings indflydelse på tilværelsen, kan kineseren være af en anden opfattelse.
\\
Følgende spørgsmål kunne være relevante:
\begin{enumerate}
    \item Hvad synes du om at data om dig gemmes?
    \item Ved du hvordan data anonymiseres?
    \item Har du tænkt over, om du kan tale fortroligt derhjemme, når telefonen formentlig lytter med?
    \item Listes teknologier, som overvåger dig, ind i dit liv, uden du er klar over det? (Coronapas, NemID, eKørekort på telefonen)
    \item Er der funktionalitet, f.eks. i din telefon, som du ikke vil undvære, selvom data om dig gemmes?
\end{enumerate}
\end{document}
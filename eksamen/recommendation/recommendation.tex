\documentclass{article}
\usepackage[danish]{babel}
\usepackage{graphicx}
\usepackage{outlines}
\usepackage{amssymb}
\usepackage{enumitem}
\usepackage{listings}
\usepackage{tabularx}
\usepackage{multirow}
\usepackage{listings}
\usepackage[T1]{fontenc}
\usepackage{inconsolata}
\usepackage{color}
\usepackage{hyperref}
\renewcommand{\labelitemi}{$\bullet$}
\renewcommand{\labelitemii}{$\circ$}
\renewcommand{\labelitemiii}{$\blacksquare$}
\renewcommand{\labelitemiv}{$\star$}
\definecolor{pblue}{rgb}{0.13,0.13,1}
\definecolor{pgreen}{rgb}{0,0.5,0}
\definecolor{pred}{rgb}{0.9,0,0}
\definecolor{pgrey}{rgb}{0.46,0.45,0.48}
\lstset{language=Java,
  showspaces=false,
  showtabs=false,
  breaklines=true,
  showstringspaces=false,
  breakatwhitespace=true,
  commentstyle=\color{pgreen},
  keywordstyle=\color{pblue},
  stringstyle=\color{pred},
  basicstyle=\ttfamily,
  moredelim=[il][\textcolor{pgrey}]{$$},
  moredelim=[is][\textcolor{pgrey}]{\%\%}{\%\%}
}
\title{%
Anbefalinger til artiklen \\ 
'Containerization As Learning Tool' \\
skrevet af Henning W.
}
\author{Claus Kramath}
\begin{document}
\maketitle
\thispagestyle{empty}
\newpage
\tableofcontents
\thispagestyle{empty} 
\newpage
\section{Forord}
\paragraph{}
Efter at have læst artiklen, som, efter min mening, rammer en ganske relevant problemstilling, såvel i undervisningsøjemed som i det professionelle arbejdsliv, vil jeg her blot komme med tre anbefalinger, som kan være med til at tydeliggøre overfor eventuelle beslutningstagere, hvorfor artiklen har sin berettigelse.\\
\section{Anbefalinger}
\paragraph{Eksempelsammenligning}
For at klæde beslutningstagere på til at træffe oplyste valg, kunne det være befordrende med et - endnu - tydeligere eksempel på forskellen i arbejdsgangene med og uden docker. \\
F.eks. kunne det beskrives hvorledes vi, i undervisningen, blev instrueret i at kopiere 3 neo4j mapper og konfigurere dem i hånden, således at der kunne opsættes et såkaldt cluster. Medtag gerne alle skridtene!\\
Til sammenligning kunne den tilsvarende docker-compose fil vises, som opstiller samme cluster automatisk - her kunne det også være relevant at nævne, at docker-compose filen også vil fejle automatisk, hvis den er konfigureret forkert eller mangelfuldt.
\paragraph{Erhvervsfaglig relevans}
Da skolen har til formål at uddanne mennesker til erhvervslivet, kunne det være relevant at inddrage data fra jobopslag, som indikerer hvor ofte kendskab til docker efterspørges.\\
Jeg har til dette formål fundet følgende graf som viser trends vedr. docker i jobopslagene på jobindex\footnote{\url{https://www.jobindex.dk/jobsoegning/analyse?q=docker\&address=\&subid=1\&jobage=\&mindate=\&maxdate=\&supid=1\&page=}}:
\begin{figure}[htb]
    \centering
    \includegraphics[width=11cm, keepaspectratio]{docker_statistik.png}    
    \label{fig:docker}
\end{figure}
\paragraph{Tendenser i erhvervet}
For at tilføje pondus, kunne man gøre klart, hvilke tendenser der findes i erhvervslivet vedr. docker eller containerisation i almindelighed.\\
Eksempelvis har Danmark fået sin første Docker Level 1 certificerede partner\footnote{\url{https://www.peytz.dk/technology-blog/foerste-docker-partner-i-danmark}} for nyligt.
\end{document}